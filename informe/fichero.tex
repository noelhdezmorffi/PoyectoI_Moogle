\documentclass{article}

\usepackage[left=2.5cm, right=2.5cm, top=3cm, bottom=3cm]{geometry}

% Para insertar imágenes
\usepackage{graphicx}

% Para agregar links
\usepackage{url}

\usepackage{amsmath, amsthm, amssymb}


\begin{document}

\title{\Huge MoogleEngine}
\author{\Large Noel Hernández Morffi, C113}
\date{\Large Julio, 2023}
\maketitle

% (resumen)
\begin{abstract}
    Motor sencillo de búsqueda en el que la lógica del ranking de relevancia es TF-IDF; las consultas tienen operadores de aparición;
     y los documentos sólo se cargan con la primera búsqueda de cada ejecución, por lo que es eficiente.
\end{abstract}

\newpage
\tableofcontents

\newpage
\section{Ejecutando el Proyecto}\label{sec:ejec}
Para ejecutar el proyecto:
\begin{enumerate}
    \item En Moogle.cs, cambiar el valor de la variable 'ContentPath' por la ruta de los archivos txt. 
    \item Cargar el proyecto: Pararse en la carpeta del proyecto, abrir un terminal y usar el comando: 'dotnet build'.
    \item Abrir la carpeta del proyecto con vscode (en la carpeta del proyecto, abrir un terminal y usar el comando: '. code')
    (o click derecho-$\rangle$ abrir con-$\rangle$  vscode).
    \item Dentro de vscode, abrir un nuevo terminal, y ejecutar el proyecto con el 
    comando: 'dotnet watch run --project MoogleServer'.
    \item Cuando abra el navegador, usar el buscador. 
\end{enumerate}

\section{Funcionalidades básicas}\label{sec:func}
Funcionalidades básicas:
\begin{itemize}
    \item El ranking de relevancia es TF-IDF.
    \item Las consultas tienen operadores de aparición:
    \begin{itemize}
        \item[-] Un símbolo '!' delante de una palabra indica que esa palabra no debe aparecer en ningún documento que sea
         devuelto.
        \item[-] Un símbolo '\textasciicircum' delante de una palabra indica que esa palabra sí debe aparecer en cualquier documento que 
        sea devuelto.
    \end{itemize}
  \item Los documentos solo se cargan con la primera búsqueda de cada ejecución, por lo que es eficiente.
\end{itemize}

\newpage
\section{Implementación}\label{sec:impl}

\subsection{Abstracción de Datos}\label{sub:abstDat}
  La clase 'Moogle' tiene variables estáticas (son estáticas porque la mayoría tiene valores constantes que no cambian 
  durante la ejecución, y almacenan información importante que se comparte entre varias funciones de la clase):
   \begin{itemize}
    \item 'contentPath' de tipo string: aquí va la ruta de la carpeta de todos los documentos a buscar.
    \item 'filePaths' de tipo array de string: en cada posición se guarda la ruta de cada archivo de extensión .txt.
    \item 'cantTxts' de tipo int: cantidad de txts.
    \item 'primeraBusqueda' de tipo bool: indica si es la primera búsqueda (consulta) o no.
    \item 'd\_TitleText' de tipo Dictionary$\langle string, string\rangle$ : diccionario en el cual cada Llave será el título de cada 
    documento; y cada Valor el texto de ese documento (para el snippet).
   \item 'd\_TitlePalabraTf' de tipo Dictionary$\langle string, Dictionary\langle string,int\rangle \rangle$ : diccionario en el cual cada 
   Llave será el título de cada documento; y cada Valor un diccionario, en el cual cada Llave será cada palabra de ese documento,
    y cada Valor la frecuencia de esa palabra (tf) en ese documento.
   \item 'd\_PalabraIdf' de tipo Dictionary$\langle string, double\rangle$ : diccionario en el cual cada Llave será cada palabra de la 
   consulta (query), y cada Valor la frecuencia inversa de esa palabra (idf).
  \end{itemize}

\subsection{Flujo de ejecución}\label{sub:fluj}
  En cada búsqueda(o llamada al método 'Moogle.Query'):
\begin{itemize}
    \item Las palabras de la consulta (query) se guardan en un array de tipo string 'palabrasQuery', que además tiene 2 
    máscaras de tipo bool ('palabrasNoDebenAparecer' y 'palabrasSiDebenAparecer') para indicar cuáles palabras tienen 
    operadores de aparición. El método 'OperadoresAparicion' se encarga de darle valores correctos a las 2 máscaras y 
    limpiar (eliminar) los símbolos de operadores de cada palabra de 'palabrasQuery'.
    \item Se crea el array de SearchItem 'items', que luego será pasado como parámetro al objeto 'result' de tipo 
    SearchResult, que devuelve el método 'Query'. Además se inicializa cada objeto de tipo SearchItem de 'items' 
    (para permitir indexar en 'items'), llamando al método 'SearchResult.Inicializar'.
   \item Si es la primera búsqueda(consulta), se cargan los diccionarios 'd\_TitleText' y 'd\_TitlePalabraTf' con sus 
   valores correctos, llamando a los métodos 'Cargar\_d\_TitleText' y 'Cargar\_d\_TitlePalabraTf' respectivamente. 
   Si no es la primera búsqueda, entonces no se cargarán estos 2 diccionarios, pues ya estarían cargados. 
    \item Se carga 'd\_PalabraIdf' llamando al método 'Cargar\_d\_PalabraIdf'. Para cada palabra del query, su
     idf = \(\log_2 \left( \frac{{\text{{cantTxts}} +1}}{{\text{{CantTxtsConPalabraX(palabra)}} +1}} \right)\).
    \item Se llama al método 'AsignarTitleSnippetScore', donde se asignan los valores de title, snippet y score a cada
     elemento de 'items', recorriendo cada elemento de 'items' a la par que cada elemento de 'd\_TitlePalabraTf'. 
     En este método se llama al método 'ScoreAndValidez', el cual además de retornar el valor correcto de 
     score (multiplicando $tf\times idf$), aprovecha el recorrido de 'items', 'd\_TitlePalabraTf' y 'palabrasQuery' para 
     dar la validez de cada documento(elemento de 'items') según las palabras que deben o no aparecer. Si el doc. no 
     es válido, title y snippet serán string vacíos, y score -1.
    \item Se crea el objeto de tipo SearchResult 'result' a devolver. Se ordena por score con el método 'OrdenaPorScore'
     de la clase SearchResult (bubble sort).
    \item Y antes de retornar el resultado, cambiamos 'primeraBusqueda' a false, pues ya se ha realizado la búsqueda(consulta).
\end{itemize}

\end{document}
